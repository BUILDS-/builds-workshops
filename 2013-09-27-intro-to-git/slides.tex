\documentclass{beamer}

\title % (optional, only for long titles)
{Introduction to Git and Github}
\subtitle{}
\author[Woodall, Chris] % (optional, for multiple authors)
{Chris~Woodall [cwoodall@bu.edu]}
\institute[BUILDS] % (optional)
{
  \inst{1}%
  BUILDS\\
  Boston University
}
\date[BUILDS 2013] % (optional)
{BUILDS Tutorial Series, Fall 2013}
\subject{Version Control}
\usetheme{Rochester}
\begin{document}

  \frame{\titlepage}
  \begin{frame}
    \frametitle{What is Git?}
    \begin{itemize}
    	\item An unpleasant or contemptible person.
    	\item A distributed revision control and source code management.
    	\item Originally developed by Linus Torvalds (``creator'' of Linux) for kernel code management
		\item Version control is any practice that tracks and provides control over changes to source code [Wikipedia, Version Control]
		\item Other version control systems: Mercurial (hg), SVN, CVS, and GNU bazaar.
		\item BUILDS uses Git and Github. 
		\begin{itemize}
			\item For example, the builds.cc website is hosted on github and
			instead of using a blog engine, like Wordpress, we use git to push posts and updates.
		\end{itemize}
    \end{itemize}
  \end{frame}

  \begin{frame}
    \frametitle{Demo \#1}
    \framesubtitle{The basics. Now follow along!}

    \begin{itemize}
	    \item \texttt{git init}
	    \item \texttt{git add}
	    \item \texttt{git status}
	    \item \texttt{git commit}
    \end{itemize}
  \end{frame}

  \begin{frame}
    \frametitle{Demo \#2}
    \framesubtitle{Branching}

    \begin{itemize}
	    \item \texttt{git branch}
	    \item \texttt{git checkout}
	    \item \texttt{git merge}
    \end{itemize}
  \end{frame}

  \begin{frame}
    \frametitle{Github}
    \framesubtitle{What is Github anyway?}

    \begin{itemize}
  		\item Adds a ``social'' aspect to git.
  		\item Acts as a place to dump your code, let other people download it, improve on it, etc.
  		\item Free for Open Source Projects.
  		\item You can make groups (like ``BUILDS-'') for managing betweens people.
  		\item Can fork and create pull requests.
	    \item Go to \url{http://github.com}
    \end{itemize}
  \end{frame}

  \begin{frame}
    \frametitle{Demo \#3}
    \framesubtitle{Gitting things from the cloud.}

    \begin{itemize}
	    \item \texttt{git clone}
	    \item \texttt{git push}
	    \item \texttt{git pull}
	    \item \texttt{git remote} (brief)
    \end{itemize}
  \end{frame}

  \begin{frame}
    \frametitle{Demo \#4}
    \framesubtitle{Making a Pull Request, or, Working Together}

    \begin{itemize}
	    \item Go to \url{http://www.github.com/BUILDS-/git-tutorial-demo}

	    \item Fork it!
	    \item Clone it!
	    \item Edit it!
	    \item Commit it and push it!
	    \item Submit a pull request for it!
    \end{itemize}
  \end{frame}

  \begin{frame}
    \frametitle{Eng-Git}
    \framesubtitle{Want to be private? or have ITAR problems?}

    \begin{itemize}
  		\item Michael Abed created \url{http://eng-git.bu.edu}
  		\item Requires you be on the BU network
  		\item Uses SSH public keys (Github also has this feature)
  		\item Free public or private repos with some nice features
  		\item Best for private or ITAR compliant projects that you don't want 
  			  to leave BU.
    \end{itemize}
  \end{frame}

  \begin{frame}
    \frametitle{Questions?}
    Slides can be found at \url{https://github.com/BUILDS-/builds-workshops}
  \end{frame}
\end{document}
